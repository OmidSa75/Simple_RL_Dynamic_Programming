\documentclass{article}
\usepackage[left=2cm, right=3cm, top=2cm, bottom=2cm]{geometry}

\usepackage{float}
\usepackage{graphicx}
\usepackage{xcolor}
\usepackage{listings}

\usepackage{fancybox}
\definecolor{codegreen}{rgb}{0,0.6,0}
\definecolor{codegray}{rgb}{0.5,0.5,0.5}
\definecolor{codepurple}{rgb}{0.58,0,0.82}
\definecolor{backcolour}{rgb}{0.95,0.95,0.92}

\lstdefinestyle{mystyle}{
	backgroundcolor=\color{backcolour},   
	commentstyle=\color{codegreen},
	keywordstyle=\color{magenta},
	numberstyle=\tiny\color{codegray},
	stringstyle=\color{codepurple},
	basicstyle=\ttfamily\footnotesize,
	breakatwhitespace=false,         
	breaklines=true,                 
	captionpos=b,                    
	keepspaces=true,                 
	numbers=left,                    
	numbersep=5pt,                  
	showspaces=false,                
	showstringspaces=false,
	showtabs=false,                  
	tabsize=4
}
\lstset{style=mystyle}

\usepackage{xepersian}
\settextfont[]{XB Yas}

\title{
حل مسئله‌ی راهروی مارپیچ با برنامه‌سازی پویا
}

\author{\\امید صادق‌نژاد\\

9912654}
\date{}

\begin{document}
	
	\maketitle
	\hrule
	\bigskip
	
	نتایج کد پیاده‌سازی شده برای مسئله‌ی راهروی مارپیچ به صورت زیر می‌باشد. حرکت ها به این صورت می‌باشند
	\\
	\lr{
\{0:up, 1:down, 2:right, 3:left\}	
}
	\\
	
	\bigskip
	
	برای حالتی که هر خانه امتیاز 
	\lr{-0.04}
	باشد:
	مقادیر سیاست‌ها و ارزش هر خانه در ابتدای الگوریتم به صورت زیر است.
	\begin{latin}
		\begin{lstlisting}
		[ 1.  2.  2.  1.]
		[ 1. nan  0.  2.]
		[ 0.  0.  2.  1.]]
		
		[[0.63945675 0.71107754 0.62066005 0.         ]
		[0.87689969        nan  0.66123633 0.         ]
		[0.51633973 0.53541164  0.41536241 0.51248974]]
		\end{lstlisting}
	\end{latin}

این مقادیر در انتهای آموزش الگوریتم به صورت زیر می‌باشد.
	\begin{latin}
	\begin{lstlisting}
	   [[0.92  0.96  1.    0. ]
		[0.88  nan   0.96  0. ]
		[0.84  0.88  0.92 0.88]]
	
       [[ 2.  2.   2.   1.]
		[ 0. nan   0.   2.]
		[ 2.  2.   0.   3.]]
	\end{lstlisting}
\end{latin}
	
	
	برای حالتی که هر خانه امتیاز 
	\lr{-4}
	باشد:
	
		\begin{latin}
		\begin{lstlisting}
		[[ 1.  0.  2.  2.]
		[ 1. nan  0.  1.]
		[ 1.  1.  2.  1.]]
		
		[[0.45017349 0.49564106 0.84728801 0.        ]
		[0.39408231        nan 0.16206914 0.        ]
		[0.60292376 0.46087325 0.50513331 0.53498239]]
		\end{lstlisting}
	\end{latin}
	
	این مقادیر در انتهای آموزش الگوریتم به صورت زیر می‌باشد.
	\begin{latin}
		\begin{lstlisting}
		[[ -7.  -3.   1.   0.]
		[-11.  nan  -1.   0.]
		[-13.  -9.  -5.  -1.]]
		
		[[ 2.  2.  2.  2.]
		[ 0. nan  2.  1.]
		[ 2.  2.  2.  0.]]
		\end{lstlisting}
	\end{latin}
در حالت دوم عامل سعی دارد سریع‌تر به پایان برسد.
	
\end{document}